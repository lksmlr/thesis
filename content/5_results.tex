\chapter{Ergebnisse}\label{ch:results}


\section{Ergebnisse der Basis-Modelle}\label{sec:results_basemodels}


\begin{figure}[h]
    \centering
    \begin{tikzpicture}
        \begin{axis}[
            ybar,
            bar width=20pt,
            width=10cm, height=7cm,
        % --- STIL & FARBEN ---
            axis lines*=left,
            y axis line style={opacity=0},
            ymajorgrids=true,
            grid style={gray!20, dashed},
        % --- ACHSEN ---
            ymin=0, ymax=1.0,
            ylabel={Score},
            ylabel style={font=\sffamily\bfseries, yshift=5pt},
            symbolic x coords={KG5b, Vertraege, Sonstiges},
            xtick=data,
            xticklabels={KG5b, Verträge, Sonstiges},
            xtick style={draw=none},
            xticklabel style={font=\sffamily, yshift=2pt},
            yticklabel style={font=\sffamily},
        % --- LEGENDE ---
            legend style={at={(0.5,-0.15)}, anchor=north, legend columns=-1, draw=none, fill=none, font=\sffamily},
        % --- BESCHRIFTUNG ---
            nodes near coords,
            nodes near coords style={font=\sffamily\footnotesize},
            enlarge x limits=0.25
        ] % <--- WICHTIG: Hier endet die Konfiguration, erst danach darf eine Leerzeile kommen

            % Plot 1
            \addplot[draw=none, fill=rred, nodes near coords style={color=rred}] coordinates {
                (KG5b,0.82)
                (Vertraege,0.59)
                (Sonstiges,0.80)
            };

            % Plot 2
            \addplot[draw=none, fill=oorange, nodes near coords style={color=oorange}] coordinates {
                (KG5b,0.46)
                (Vertraege,0.50)
                (Sonstiges,0.80)
            };

            \legend{Qwen-2.5-VL-7b, Pixtral-12B}
        \end{axis}
    \end{tikzpicture}
    \caption{Vergleich der Modellleistungen nach Dokumentenart}
    \label{fig:result_basemodels}
\end{figure}


\section{Ergebnisse des Fine-Tunings}\label{sec:results_fine_tuning}

\section{Ergebnisse YOLO/OCR-Pipeline}\label{sec:results_yolo_ocr_pipeline}

\section{Ergebnisse im Überblick}\label{sec:results_overview}


