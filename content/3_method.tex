\chapter{Methodik und Datenaufbereitung}\label{ch:data}

\section{Modellierung der Dokumentenarten als JSON}\label{sec:documents_as_json}

\section{Evaluation}\label{sec:evaluation}

\section{Datenvorbereitung}\label{sec:preprocessing}

\subsection{Testdatensatz}\label{subsec:test_dataset}

\subsection{Trainingsdatensatz}\label{subsec:train_dataset}

\section{Modellauswahl}\label{sec:modelselection}

\section{Fine-Tuning}\label{sec:fine_tuning}



\section{Dokumentenarten}\label{sec:documents1}

Ziel der Arbeit ist die \gls{IE} aus den genannten Dokumentenklassen, sowie deren Klassifikation.
Um die gewonnenen Informationen aus dem Dokument bereitzustellen, wird ein flaches \gls{JSON}-Schema verwendet.
Die Wahl dieses Formats wird dadurch begünstigt, dass moderne \gls{LLM}s durch ihr Training bereits eine hohe Zuverlässigkeit in der Generierung valider \gls{JSON}-Strukturen aufweisen.

Die Qualität der Dokumente, die im Onlineportal hochgeladen werden, ist sehr unterschiedlich.
Neben Scans mit guter Belichtung und hoher Auflösung enthält der Datensatz auch Fotos, die aus verschiedenen Winkeln und Entfernungen aufgenommen wurden.
Eine zusätzliche Herausforderung ist, dass die Dokumente häufig handschriftlich ausgefüllt sind.

In den Test- und Trainingsdatensätzen kommen diese Probleme in unterschiedlichen Konstellationen vor.
Der Testdatensatz umfasst insgesamt 60 Dokumente, die sich gleichmäßig auf 20 Ausbildungsverträge, 20 KG5b-Formulare und 20 sonstige Dokumente verteilen.
Der Trainingsdatensatz setzt sich aus 165 KG5b-Formularen, 227 Ausbildungsverträgen und 218 sonstigen Dokumenten zusammen.

Die Erstellung Ground Truth erfolgte in einem zweistufigen Verfahren.
Zunächst wurden die \gls{JSON}-Strukturen für den Testdatensatz vollständig manuell erstellt.
Dieser Testdatensatz diente anschließend dazu, den System-Prompt zu optimieren, bis die Ergebnisse des Basis-Modells eine zufriedenstellende Qualität erreichten.
Um den Annotationsaufwand für den Trainingsdatensatz zu reduzieren, wurde dieser optimierte Prompt für ein sogenanntes Pre-Labeling genutzt.
Das Modell generierte dabei erste Vorschläge für die \gls{JSON}-Schemas, welche im Anschluss manuell validiert und korrigiert wurden.
Dieser modellgestützte Annotationsprozess (Model-Assisted Labeling) ermöglichte eine effiziente Erstellung der Trainingsdaten bei gleichbleibend hoher Datenqualität.


\subsection{KG5b}\label{subsec:kg5b1}

Da es sich um ein standardisiertes Formular handelt, ist das Layout gleichbleibend.
Theoretisch vereinfacht dies die Extraktion der Informationen, da relevante Felder an denselben Positionen zu erwarten sind.
In der Praxis ergibt sich jedoch eine hohe Varianz durch den Ausfüllprozess.
Obwohl der Vordruck digital ausfüllbar angeboten wird, erfolgt die Bearbeitung in der Regel handschriftlich.
Zudem erzwingen Unterschriften und Firmenstempeln einen doppelten Medienbruch.
Das Dokument wird ausgedruckt, bearbeitet und anschließend wieder digitalisiert.

Das Zielformat für die Extraktion wird durch das folgende \gls{JSON}-Schema definiert:

\lstinputlisting[language=Python, caption={JSON-Schema für KG5b}, captionpos=b, label={lst:kg5b_schema}]{listings/kg5b.json}

Das Feld \texttt{type} stellt die Klassifikation dar, wobei die restlichen Felder die Information Extraction abbilden.

Eine besondere Herausforderung bei der Verarbeitung stellen die binären Informationen dar, wie etwa die Checkbox für das Feld \texttt{apprenticeship\_ended}.
Hier muss das Modell visuell unterscheiden, ob ein Kasten leer, angekreuzt oder durchgestrichen ist.
Ebenso kritisch ist die Detektion von \texttt{signature\_company} und \texttt{stamp\_company}.
Im Gegensatz zu Textfeldern ist hier nicht der textliche Inhalt des Stempels relevant, sondern lediglich dessen Vorhandensein.
Die Stempel sind oft blass oder eingefärbt, was besonders bei Schwarz-Weiß Scans zu einer schlechten Qualität führt.
Zudem liegen die Felder \texttt{date\_document} und \texttt{signature\_company} nah beieinander.
Unterschriften sind regelmäßig größer als ihr vorgesehener Platz, wodurch das Feld \texttt{date\_document} überdeckt wird.
Zusätzlich werden die Stempel häufig zusammen mit der Unterschrift auf der linken Seite platziert.



\newpage
\subsection{Ausbildungsvertrag}\label{subsec:vertraege1}



Eine weitere Herausforderung stellt der Umfang der Dokumente dar.
Ausbildungsverträge umfassen häufig mehrere Seiten, wobei relevante Informationen und irrelevante Informationen, wie zum Beispiel gesetzliche Rahmenbedingungen, AGBs oder Rechtsbelehrungen, gemischt auftreten.
Dies erschwert die Fokussierung des Modells auf die relevanten Daten.

Analog zu den KG5b-Formularen liegt auch hier durch die notwendigen Unterschriften und Stempel ein Medienbruch vor.
Ein wesentlicher Vorteil gegenüber den handschriftlich ausgefüllten Anträgen besteht jedoch darin, dass die inhaltlichen Daten der Verträge fast ausschließlich maschinenschriftlich vorliegen, was die Zeichenerkennung erleichtert.


Das Zielformat für die Extraktion wird durch das folgende \gls{JSON}-Schema definiert:

\lstinputlisting[language=Python, caption={JSON-Schema Vertrag}, captionpos=b,label={lst:lstvertrag}]{listings/vertrag.json}

Die Felder stimmen weitestgehend mit denen der KG5b-Formulare überein, jedoch entfallen die Felder \texttt{exam\_month} und \texttt{apprenticeship\_ended}, da diese selten auf Verträgen zu finden sind.


\subsection{Sonstige Dokumente}\label{subsec:other_documents1}

Die Kategorie \texttt{Sonstiges} steht als Auffangklasse für alle restlichen Dokumente bereit.
Hier befinden sich zum Beispiel Schulbescheinigungen, Anträge auf Eintragung bei der Handelskammer oder Studienbescheinigungen.
Da diese Dokumente keine relevanz für die Weiterbeantragung des Kindergeldes bei volljährigen Auszubildenden hat, werden keine Information benötigt.
Somit ist rein die Klassifizierung wichtig, weshalb das extrahierte \gls{JSON}-Schema folgendermaßen aussieht:

\lstinputlisting[language=Python, caption={JSON-Schema Sonstiges}, captionpos=b,label={lst:lstsonstiges}]{listings/sonstiges.json}


\section{Test- und Trainingsdatensatz}\label{sec:dataset}




