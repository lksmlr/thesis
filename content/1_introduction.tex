\chapter{Einleitung}\label{ch:introduction}

\section{Motivation}\label{sec:motivation}

Durch das erhöhte Aufkommen an Kindergeldanträgen wird eine teilautomatisierte Erkennung der Dokumente immer relevanter.
Im Zusammenarbeit mit der Familienkasse (\gls{FamKa}) wurde ein System entwickelt, dass verschiedene Arten von Ausbildungsbescheinigungen erkennen kann.
Dieses System besteht aktuell aus mehreren OCR- und YOLO-Modellen und wurde erst im Oktober 2025 in Produktion gesetzt.
\newline
Durch die rasante Entwicklung von Large Language Models wurden diese schnell interessant für eine solche Aufgabe.
Insbesondere Multimodal Large Language Models, die zusätzlich einen Vision-Encoder besitzen, könnte Probleme der ursprünglichen Pipeline lösen.

\section{Problemstellung}\label{sec:problem}

Um einen Kindergeldantrag zu bewilligen, werden verschieden Information benötigt.
Dazu gehören zum Beispiel der Name des Kindes, der Name des Kindergeldberechtigten und das Start- und Enddatum der Ausbildung.
Dokumente, die handschriftlich ausgefüllt wurden, stellen eine besondere Herausforderung für die Erkennung dar.
Hier stoßen die OCR-Modelle häufig an ihre Grenzen und die erkannten Informationen sind falsch.
Zudem ist die häufig sehr unterschiedliche Struktur ein großes Problem, da es für solche Fälle wenige Trainingsdaten vorhanden sind.

\section{Zielsetzung}\label{sec:goal}

- Forschungsfragen

% TODO: Hier muss die Section Zielsetzung noch geschrieben werden.

