\chapter{Methodik und Datenaufbereitung}\label{ch:data}


\section{Modellierung der Dokumentenarten als JSON}\label{sec:documents_as_json}

Um eine standardisierte Weiterverarbeitung der Modellausgaben zu gewährleisten, ist eine feste Struktur essenziell.
Da moderne \glspl{VLM} darauf trainiert sind, Antworten im \gls{JSON}-Format zu liefern, werden die extrahierten Informationen in ein vordefiniertes \gls{JSON}-Schema überführt.
Diese Schemata unterscheiden sich je nach Dokumentenart, da jeweils unterschiedliche Informationen relevant sind.


\subsection{KG5b}\label{subsec:json_kg5b}

Das Schema für das Formular KG5b ist in Abbildung~\ref{fig:json_kg5b} definiert.
Die Klassifikation spiegelt sich im Feld \texttt{type} wider, während die restlichen Felder Platzhalter für die \gls{IE} darstellen.

\begin{figure}[!ht]
    \centering
    \includegraphics[width=0.5\textwidth]{figures/kg5b_json}
    \caption{JSON-Schema des Dokumententyps KG5b}
    \label{fig:json_kg5b}
\end{figure}


\subsection{Ausbildungsvertrag}\label{subsec:json_vertrag}

Das Schema für die Ausbildungsverträge ist in Abbildung~\ref{fig:json_vertrag} definiert.
Wie im Schema des KG5b dient das Feld \texttt{type} der Klassifikation, während die restlichen Felder die \gls{IE} repräsentieren.

\begin{figure}[!ht]
    \centering
    \includegraphics[width=0.5\textwidth]{figures/vertrag_json}
    \caption{JSON-Schema des Dokumententyps Ausbildungsvertrag}
    \label{fig:json_vertrag}
\end{figure}


\subsection{Sonstige Dokumente}\label{subsec:json_sonstiges}

Das Schema für die nicht relevanten Dokumente ist in Abbildung~\ref{fig:json_sonstiges} definiert.
Da hier keine Informationen benötigt werden, sondern rein die Klassifikation von Nutzen ist, besteht das Schema lediglich aus dem Feld \texttt{type}.

\begin{figure}[!ht]
    \centering
    \includegraphics[width=0.5\textwidth]{figures/sonstiges_json}
    \caption{JSON-Schema des Dokumententyps Sonstiges}
    \label{fig:json_sonstiges}
\end{figure}


\section{Evaluation}\label{sec:evaluation}

\subsection{Vergleichslogik der unterschiedlichen JSON-Felder}\label{subsec:comparator}

\subsection{Bewertung der Klassifikation}\label{subsec:evaluation_classification}

\subsection{Bewertung der Information Extraction}\label{subsec:evaluation_ie}

\section{Datenvorbereitung}\label{sec:preprocessing}

\subsection{Testdatensatz}\label{subsec:test_dataset}

\subsection{Trainingsdatensatz}\label{subsec:train_dataset}

\section{Modellauswahl}\label{sec:modelselection}
- Promptdesign
- warum kein one shot few shot usw
\section{Fine-Tuning}\label{sec:fine_tuning}
