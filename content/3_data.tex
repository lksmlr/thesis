\chapter{Datenbasis}\label{ch:data}

\section{Dokumentenarten}\label{sec:documents}


Im Rahmen der Kindergeldbeantragung sind verschiedene Nachweise gültig.
Zu den anerkannten Dokumententypen zählen der offizielle Vordruck der Bundesagentur für Arbeit (KG5b)\cite{KG5b} sowie Ausbildungsverträge.
Zusätzlich laden Kunden häufig weitere Unterlagen, wie beispielsweise Schulbescheinigungen, im Portal hoch.
Da diese für den Kindergeldantrag nicht im Fokus stehen, werden sie im Folgenden unter der Kategorie „Sonstiges“ zusammengefasst.

Ziel der Arbeit ist die Extraktion spezifischer Informationen aus den genannten Dokumentenklassen.
Um die gewonnenen Informationen aus dem Dokument bereitzustellen, wird ein flaches \gls{JSON}-Schema verwendet.
Die Wahl dieses Formats wird dadurch begünstigt, dass moderne \gls{LLM}s durch ihr Training bereits eine hohe Zuverlässigkeit in der Generierung valider \gls{JSON}-Strukturen aufweisen.

Die Qualität der Dokumente, die im Onlineportal hochgeladen werden, ist sehr unterschiedlich.
Neben Scans mit guter Belichtung und hoher Auflösung enthält der Datensatz auch Fotos, die aus verschiedenen Winkeln und Entfernungen aufgenommen wurden.
Eine zusätzliche Herausforderung ist, dass die Dokumente häufig handschriftlich ausgefüllt sind.

Die Test- und Trainingsdatensätze bestehen aus unterschiedlichen Konstellationen dieser Probleme.
Der Testdatensatz besteht aus insgesamt 60 Dokumenten, wobei es 20 Verträge, 20 KG5bs und 20 sonstige Dokumente gibt.

\subsection{KG5b}\label{subsec:kg5b}

Wie bereits erwähnt ist ein KG5b-Dokument ein offizielles Dokument der Bundesagentur für Arbeit.
Hierbei können volljährige Kinder ihr aktuelles Ausbildungsverhältnis gegenüber der Kindergeldstelle bekannt geben.



\subsection{Ausbildungsvertrag}\label{subsec:vertraege}

\subsection{Sonstige Dokumente}\label{subsec:other_documents}

\section{Datenschutz}\label{sec:data_privacy}




