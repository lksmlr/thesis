\chapter{Methodik und Datenaufbereitung}\label{ch:data}


\section{Modellierung der Dokumentenarten als JSON}\label{sec:documents_as_json}

Um eine standardisierte Weiterverarbeitung der Modellausgaben zu gewährleisten, ist eine feste Struktur essenziell.
Da moderne \glspl{VLM} darauf trainiert sind, Antworten im \gls{JSON}-Format zu liefern, werden die extrahierten Informationen in ein vordefiniertes \gls{JSON}-Schema überführt.
Diese Schemata unterscheiden sich je nach Dokumentenart, da jeweils unterschiedliche Informationen relevant sind.


\subsection{KG5b}\label{subsec:json_kg5b}

Das Schema für das Formular KG5b ist in Abbildung~\ref{fig:json_kg5b} definiert.
Die Klassifikation spiegelt sich im Feld \texttt{type} wider, während die restlichen Felder Platzhalter für die \gls{IE} darstellen.

\begin{figure}[!ht]
    \centering
    \includegraphics[width=0.5\textwidth]{figures/kg5b_json}
    \caption{JSON-Schema des Dokumententyps KG5b}
    \label{fig:json_kg5b}
\end{figure}


\subsection{Ausbildungsvertrag}\label{subsec:json_vertrag}

Das Schema für die Ausbildungsverträge ist in Abbildung~\ref{fig:json_vertrag} definiert.
Wie im Schema des KG5b dient das Feld \texttt{type} der Klassifikation, während die restlichen Felder die \gls{IE} repräsentieren.

\begin{figure}[!ht]
    \centering
    \includegraphics[width=0.5\textwidth]{figures/vertrag_json}
    \caption{JSON-Schema des Dokumententyps Ausbildungsvertrag}
    \label{fig:json_vertrag}
\end{figure}


\subsection{Sonstige Dokumente}\label{subsec:json_sonstiges}

Das Schema für die nicht relevanten Dokumente ist in Abbildung~\ref{fig:json_sonstiges} definiert.
Da hier keine Informationen benötigt werden, sondern rein die Klassifikation von Nutzen ist, besteht das Schema ausschließlich aus dem Feld \texttt{type}.

\begin{figure}[!ht]
    \centering
    \includegraphics[width=0.5\textwidth]{figures/sonstiges_json}
    \caption{JSON-Schema des Dokumententyps Sonstiges}
    \label{fig:json_sonstiges}
\end{figure}


\section{Evaluation}\label{sec:evaluation}

\subsection{Vergleichslogik der unterschiedlichen JSON-Felder}\label{subsec:comparator}

Um ein realistisches Ergebnis zu erhalten, müssen unterschiedliche Feldtypen spezifisch verglichen werden.
Die Menge an distinkten Typen beinhaltet: Namen, Booleans, Zeichenketten, Daten und Monate.

Namen werden mit der Levenshtein-Similarity verglichen, um kleinere Fehler oder verschiedene Varianten zu tolerieren.
Beispielsweise wird ein Name, der \texttt{ue} statt \texttt{ü} enthält, nicht als Fehler erkannt.
Des Weiteren werden minimale \gls{OCR}-Fehler gefiltert.
Die \gls{YOLO}/\gls{OCR}-Pipeline verwendet ebenso diesen Vergleich mit den gleichen Schwellenwerten, somit wird das Ergebnis vergleichbar.

Zeichenketten und Booleans werden exakt verglichen.
Hierzu zählen die Felder \texttt{type}, \texttt{stamp\_company}, \texttt{signature\_company}, \texttt{signature\_child}, \texttt{signature\_legal\_guardian} und \texttt{apprenticeship\_finished}.
Lediglich die booleschen Werte werden zu \texttt{true} normalisiert, da hier eine potenzielle Fehlerquelle liegt.

Daten werden einheitlich in das Format DD.MM.YYYY gebracht, obwohl dies nicht dem internationalen Standard entspricht.
Das ist darauf zurückzuführen, dass so die Erkennung des \glspl{VLM} am robustesten ist, da die deutschen Dokumente auch mit diesem Format ausgefüllt werden.

Monate werden in das Format MM normalisiert.
Hierbei werden ausgeschriebene Monatsnamen sowie vollständige Datumsangaben angepasst.

\subsection{Bewertung der Klassifikation}\label{subsec:evaluation_classification}

\subsection{Bewertung der Information Extraction}\label{subsec:evaluation_ie}

\section{Datenvorbereitung}\label{sec:preprocessing}

\subsection{Testdatensatz}\label{subsec:test_dataset}

\subsection{Trainingsdatensatz}\label{subsec:train_dataset}

\section{Modellauswahl}\label{sec:modelselection}
- Promptdesign
- warum kein one shot few shot usw
\section{Fine-Tuning}\label{sec:fine_tuning}
