\chapter{Methodik und Implementierung}\label{ch:data}

Im Rahmen dieses Proof of Concepts werden verschiedene \glspl{VLM} vergleichend gegenüber der bestehenden \gls{YOLO}/\gls{OCR}-Pipeline evaluiert.
Für die Entscheidung, ob und welches \gls{VLM} die bisherige Pipeline ersetzen wird, sind sowohl die Modellleistungen als auch der Ressourcenverbrauch von zentraler Bedeutung.
Aus diesem Grund sollen die folgenden Forschungsfragen beantwortet werden:

\begin{enumerate}
    \item Inwiefern ist ein nicht angepasstes \gls{VLM} hinsichtlich der Klassifikations- und Extraktionsgenauigkeit der trainierten \gls{YOLO}/\gls{OCR}-Pipeline überlegen, und welche Vorteile bietet ein einzelnes generalistisches Modell gegenüber der Verwendung spezialisierter Einzelsysteme?
    \item Wie verhält sich ein kleineres, domänenspezifisch trainiertes Modell gegenüber einem leistungsstärkeren Basismodell in Bezug auf Performanz und Effizienz?
    \item In welchem Verhältnis steht der Ressourcenverbrauch der \glspl{VLM} zu dem der \gls{YOLO}/\gls{OCR}-Pipeline?
\end{enumerate}


\section{Modellierung der Dokumentenarten als JSON-Objekte}\label{sec:documents_as_json}

Um eine standardisierte Weiterverarbeitung der Modellausgaben zu gewährleisten, ist eine feste Struktur essenziell.
Da moderne \glspl{VLM} darauf trainiert sind, Antworten im JavaScript Object Notation ()\gls{JSON})-Format zu liefern, werden die extrahierten Informationen in ein vordefiniertes \gls{JSON}-Schema überführt.
Ein wesentlicher Vorteil gegenüber der herkömmlichen Pipeline ist die gleichzeitige Durchführung der Klassifikation sowie der \gls{IE} in einem einzigen Inferenzschritt.
Das Modell klassifiziert die Dokumentenart, die im Feld \texttt{type} abgebildet wird.
Während das Feld \texttt{type} in allen \gls{JSON}-Schemata konsistent vorhanden ist, variieren die Felder der \gls{IE} je nach Dokumentenart.

Bei den binären, booleschen Feldern markiert ein \texttt{true} das Vorhandensein eines Merkmals, während ein \texttt{false} dessen Fehlen oder das Nicht-Erkennen repräsentiert.
Um eine strukturelle Konsistenz zu erzwingen, werden Felder, bei denen die dazugehörige Information im Dokument fehlt, bei booleschen Typen mit \texttt{false} und bei Textfeldern mit einer leeren Zeichenkette belegt.

Da die Dokumente in den meisten Fällen aus mehreren Seiten bestehen, repräsentiert ein \gls{JSON}-Objekt das mehrseitige Dokument.
Sollten Dokumente verschiedener Arten vermischt vorliegen, generiert das Modell für jede erkannte Dokumentenart ein separates \gls{JSON}-Objekt.

Die \gls{JSON}-Schemata sind in den Abbildungen

Um den \gls{JSON}-Output der \glspl{VLM} zu limitieren, existieren verschiedene Ansätze.
Während komplexere Lösungen nur Tokens zulassen die in \gls{JSON}-Objekten enthalten sein könnten, wird in dieser Evaluation ein einfacherer Ansatz gewählt.
Hierbei werden aus der generierten Antwort \glspl{JSON} gefiltert und geparst.
Sobald das Parsen der \gls{JSON} fehlschlägt, wird diese zurück an das \gls{VLM} geliefert, mit der Anweisung diese zu korrigieren.
Das Modell bekommt insgesamt drei Versuche eine valide \gls{JSON} zu generieren, bevor das Dokument als ungültig eingestuft wird.
% TODO: Code der retry-logik und pydantic validierung + jsonparser langchain

\subsection{KG5b}\label{subsec:json_kg5b}

Das Schema für das Formular KG5b ist in Abbildung~\ref{fig:json_kg5b} definiert.

\begin{figure}[!ht]
    \centering
    \includegraphics[width=0.5\textwidth]{figures/kg5b_json}
    \caption{JSON-Schema des Dokumententyps KG5b}
    \label{fig:json_kg5b}
\end{figure}


\subsection{Ausbildungsvertrag}\label{subsec:json_vertrag}

Das Schema für die Ausbildungsverträge ist in Abbildung~\ref{fig:json_vertrag} definiert.

\begin{figure}[!ht]
    \centering
    \includegraphics[width=0.5\textwidth]{figures/vertrag_json}
    \caption{JSON-Schema des Dokumententyps Ausbildungsvertrag}
    \label{fig:json_vertrag}
\end{figure}


\subsection{Sonstige Dokumente}\label{subsec:json_sonstiges}

Das Schema für die nicht relevanten Dokumente ist in Abbildung~\ref{fig:json_sonstiges} definiert.
Da hier keine Informationen benötigt werden, sondern rein die Klassifikation von Nutzen ist, besteht das Schema ausschließlich aus dem Feld \texttt{type}.

\begin{figure}[!ht]
    \centering
    \includegraphics[width=0.3\textwidth]{figures/sonstiges_json}
    \caption{JSON-Schema des Dokumententyps Sonstiges}
    \label{fig:json_sonstiges}
\end{figure}
\section{Metriken und Validierung der VLM-Ausgabe}\label{sec:validation}

\section{Evaluation und Auswahl des Basismodells}\label{sec:model_selection}

\section{Modell-Optimierung durch Supervised Fine-Tuning (SFT)}\label{sec:fine_tuning}
