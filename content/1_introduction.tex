\chapter{Einleitung}\label{ch:introduction}

\section{Motivation}\label{sec:motivation}

Die öffentliche Verwaltung in Deutschland steht vor einer großen Herausforderung.
Durch den demografischen Wandel verliert der öffentliche Sektor eine hohe Anzahl an erfahrenen Sachbearbeitern~\cite{DemografischerWandel}, während der Anspruch der Bürger steigt.
Dieser Wandel wird ohne Digitalisierung und Automatisierung der Prozesse kaum abzufangen sein.

Auch in der Bundesagentur für Arbeit macht sich dieser Wandel hin zu mehr Digitalisierung bemerkbar.
Ein Beispiel hierfür ist die Bearbeitung von Kindergeldanträgen.
Mit einem jährlichen Aufkommen von mehreren Millionen Anträgen~\cite{AnzahlAntraegeProJahr} und saisonalen Spitzen kommen hier die Sachbearbeiter an ihre Grenzen.

Um sich dieser Herausforderung zu stellen, startete bereits im Oktober 2025 der Produktivbetrieb eines teilautomatisierten Systems.
Der Fokus liegt hierbei auf den Kindergeldanträgen volljähriger Auszubildender.
Das System klassifiziert die Dokumente, erkennt mithilfe von \gls{OCR}- und \gls{YOLO}-Modellen Inhalte aus hochgeladenen Dokumenten und zeigt dem Sachbearbeiter entsprechende Vorschläge an.
Diese aktuelle OCR/YOLO-Pipeline optimiert die Bearbeitung der Kindergeldanträge für die Sachbearbeiter bereits, stößt jedoch an ihre Grenzen.

Aufgrund der schnellen Entwicklung im Bereich der Vision Language Models (VLMs) werden diese zunehmend relevanter für Aufgaben im Bereich der Dokumentenverarbeitung~\cite{SurveyVLMs}.
Die vorliegende Arbeit evaluiert, inwiefern ein solches Modell das Potenzial besitzt, die OCR/YOLO-Pipeline zu ersetzen.


\section{Problemstellung}\label{sec:problem}

Seitdem die aktuelle Pipeline im operativen Einsatz ist, zeigen sich verschiedene Herausforderungen.

Das Hauptproblem ist die hohe Varianz der hochgeladenen Dokumente, insbesondere der Ausbildungsverträge.
Die verschiedenen Firmen und Kammern nutzen alle unterschiedliche Layouts, wodurch das Training der Modelle komplex ist.
Es ist schwierig, die Varianz in den Trainingsdaten abzubilden, zumal es zeitaufwendig ist, eine solche Menge an Trainingsdaten bereitzustellen.

Die Dokumente, die relevant für den Kindergeldantrag bei volljährigen Auszubildenden sind, benötigen in allen Fällen eine Unterschrift.
Dadurch werden die Dokumente zwingenderweise ausgedruckt, ausgefüllt und unterschrieben, um schlussendlich wieder eingescannt zu werden.
Dieser Prozess, auch Medienbruch genannt, führt zu einer geminderten Bildqualität.

Angesichts der Verwendung vieler unterschiedlicher Modelle ist die Wartung des Systems aufwendig.
Eine Anpassung an neue Gegebenheiten ist ein zeitaufwendiger Prozess.
Schon kleine Änderungen an der Laufzeitumgebung, wie zum Beispiel eine neue Python-Version, können ein erneutes Training erforderlich machen.


\section{Zielsetzung}\label{sec:goal}

Das Ziel dieser Bachelorarbeit ist die Entwicklung und Evaluation eines prototypischen Systems zur teilautomatisierten Klassifikation und Informationsextraktion von Dokumenten.
Wegen der hohen Anzahl an manuell bearbeiteter Anträge und der Probleme der OCR/YOLO-Pipeline soll untersucht werden, inwieweit VLMs diesen Prozess effizienter gestalten können.
Die VLM-Pipeline soll den Dokumententyp eigenständig erkennen und definierte Inhalte wie Namen, Unterschriften und Stempel in ein standardisiertes Format überführen.

Ein weiteres Ziel ist der Vergleich unterschiedlicher Modellansätze hinsichtlich ihrer Extraktionsleistung und Effizienz.
Hierbei wird ein kleineres, domänenspezifisch nachtrainiertes Modell mit einem leistungsstärkeren Modell mit höherer Parameteranzahl verglichen.
Die Evaluation untersucht, ob ein Fine-Tuning mit einem begrenzten Datensatz vergleichbare oder bessere Ergebnisse erzielt als der Einsatz größerer Basismodelle.

Diese Arbeit konzentriert sich auf die Machbarkeit und die Evaluation der Modelle anhand eines Testdatensatzes.
Die Entwicklung zielt auf einen funktionsfähigen Prototyp ab, der lokal betrieben wird.
Eine vollständige Integration in das bestehende operative Fachverfahren ist nicht Gegenstand dieser Arbeit.