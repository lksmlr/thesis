\newglossaryentry{YOLO}{
    name={YOLO},
    description={
        (You Only Look Once) ist ein Deep-Learning-Verfahren zur Objekterkennung, das Objekte und die dazugehörigen Klassen in einem einzigen Durchlauf erkennt.
        Für die Detektion werden Bounding-Boxes und Klassenwahrscheinlichkeiten berechnet, was eine schnelle Erkennung ermöglicht.
    }
}

\newglossaryentry{OCR}{
    name={OCR},
    description={
        (Optical Character Recognition) ist ein Verfahren zu Texterkennung, das hand- und maschinenschriftliche Zeichen aus Bilder extrahiert und in einen Text umwandelt.
    }
}

\newglossaryentry{JSON}{
    name={JSON},
    description={
        (JavaScript Object Notation) basiert auf Schlüssel-Wert-Paaren und ist ein textbasiertes Datenformat zur strukturierten Speicherung und Übertragung von Daten.
    }
}

\newglossaryentry{VQA}{
    name={VQA},
    description={
        (Visual Question Answering) ist eine Aufgabe, bei der ein Modell ein Bild analysiert und auf Basis dessen eine Frage beantwortet.
    }
}

\newglossaryentry{Information Extraction}{
    name={Information Extraction},
    description={
        (IE) umfasst nach Jurafsky und Martin die Aufgabe, Ereignisse oder Situationen in Dokumenten zu identifizieren und die entsprechenden Felder eines vorgegebenen Templates zu füllen\cite{IE}.
        Ziel hierbei ist es, vordefinierte Entitäten aus einem Dokumentenbild zu extrahieren und diese in ein maschinenlesbares Schema zu überführen.
    }
}

\newglossaryentry{GPU}{
    name={GPU},
    description={
        (Graphics Processing Unit) ist ein spezialisierter Prozessor in einem Computer oder Server, der rechenintensive Arbeiten parallelisiert.
    }
}

\newglossaryentry{VRAM}{
    name={VRAM},
    description={
        (Video Random Access Memory) ist ein spezieller Grafikspeicher auf einer GPU, der zur schnellen Speicherung und Verarbeitung während rechenintensiver Anwendungen verwendet wird.
    }
}

\newglossaryentry{IDE}{
    name={IDE},
    description={
        (Integrated Development Environment) ist eine Entwicklungsumgebung die spezielle Werkzeuge für Entwickler bereitstellt.
    }
}

\newglossaryentry{KV-Cache}{
    name={KV-Cache},
    description={
        (Key-Value-Cache) speichert die Key-Value-Vektoren der bereits generierten Token, sodass nur die Vektoren der neuen Token berechnet werden müssen.
    }
}

\newglossaryentry{logprobs}{
    name={logprobs},
    description={
        (Logarithmic Probabilities) sind die logarithmierten Wahrscheinlichkeiten der möglichen Token pro Inferenzschritt.
    }
}

\newglossaryentry{vLLM}{
    name={vLLM},
    description={
        ist eine Open-Source-Engine für die effiziente Inferenz von großen Sprachmodellen.
    }
}

\newglossaryentry{Unsloth}{
    name={Unsloth},
    description={
        ist ein Open-Source-Framework zum beschleunigten und speichereffizienten Fine-Tuning von Large Models, das speziell für Methoden des PEFT optimiert ist.
    }
}

\newglossaryentry{Levenshtein-Similarity}{
    name={Levenshtein-Similarity},
    description={
        ist ein aus der Levenshtein-Distanz abgeleiteter, normalisierter Ähnlichkeitswert zwischen zwei Zeichenketten.
        Formal basiert sie auf der Levenshtein-Distanz, also der minimalen Anzahl von Einfügungen, Löschungen oder Ersetzungen, die nötig sind, um eine Zeichenkette in eine andere zu überführen~\cite{gld}.
    }
}

\newglossaryentry{TP}{
    name={TP},
    description={
        (True Positive) sind alle positiven Fälle, die tatsächlich als positiv klassifiziert wurden.
    }
}

\newglossaryentry{TN}{
    name={TN},
    description={
        (True Negative) sind alle negativen Fälle, die tatsächlich als negativ klassifiziert wurden.
    }
}

\newglossaryentry{FP}{
    name={FP},
    description={
        (False Positive) sind alle negativen Fälle, die fälschlicherweise als positiv klassifiziert wurden.
    }
}

\newglossaryentry{FN}{
    name={FN},
    description={
        (False Negative) sind alle positiven Fälle, die fälschlicherweise als negativ klassifiziert wurden.
    }
}

\newglossaryentry{Accuracy}{
    name={Accuracy},
    description={
        ist eine Metrik, die den Anteil der insgesamt korrekten Vorhersagen an der Gesamtzahl der getroffenen Vorhersagen beschreibt.
    }
}

\newglossaryentry{Precision}{
    name={Precision},
    description={
        gibt den Anteil der tatsächlich positiven Fälle an allen vom Modell als positiv klassifizierten Fällen an.
    }
}

\newglossaryentry{Recall}{
    name={Recall},
    description={
        gibt den Anteil der korrekt als positiv erkannten Fälle an allen tatsächlich vorhandenen positiven Fällen an.
    }
}

\newglossaryentry{F1-Score}{
    name={F1-Score},
    description={
        ist das harmonische Mittel aus Precision und Recall.
    }
}

\newglossaryentry{Latenz}{
    name={Latenz},
    description={
        ist die Zeitdauer, die das System für die Inferenz einer einzigen Antwort benötigt.
    }
}

\newglossaryentry{R-K-F-Formel}{
    name={R-K-F-Formel},
    description={
        ist eine Strategie im Prompt Engineering zur strukturierten Anweisung eines LLMs. Dabei steht R für Rolle (Rolle, die das Modell einnehmen soll), K für Kontext (relevante Hintergrundinformationen) und F für Format (Vorgabe der gewünschten Ausgabestruktur).
    }
}


\newglossaryentry{Prompt Engineering}{
    name={Prompt Engineering},
    description={
        ist ein Prozess, um die effektivste Anweisung für eine spezifische Aufgabe zu finden\cite{prompt_engineering}.
    }
}

\newglossaryentry{Zero-Shot Prompting}{
    name={Zero-Shot Prompting},
    description={
        ist eine Methode, bei der dem Modell eine Aufgabe gestellt wird, ohne dass zusätzliche Beispiele im Prompt enthalten sind.
    }
}

\newglossaryentry{One-Shot Prompting}{
    name={One-Shot Prompting},
    description={
        ist eine Technik, bei der dem Modell genau ein Beispiel für die Lösung im Prompt bereitgestellt wird.
    }
}

\newglossaryentry{Few-Shot Prompting}{
    name={Few-Shot Prompting},
    description={
        ist ein Ansatz, bei dem dem Modell mehrere Beispiele im Prompt übergeben werden.
    }
}

\newglossaryentry{OpenAI Message Format}{
    name={OpenAI Message Format},
    description={
        ist ein standardisiertes, rollenbasiertes Datenformat zur Interaktion mit Modellen, bei dem Nachrichten nach Rollen unterteilt und Inhalte als Array aus verschiedenen Typen wie Text und Bild definiert werden.
    }
}

\glsaddall