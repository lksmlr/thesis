\chapter{Einleitung}\label{ch:introduction}


\section{Motivation}\label{sec:motivation}

Die öffentliche Verwaltung in Deutschland steht vor einer der größten Entwicklungen ihrer Geschichte.
Durch den demografischen Wandel verliert der öffentliche Sektor in den kommenden Jahren eine signifikante Anzahl an erfahrenen Sachbearbeitern, während die Erwartungshaltung der Bürger an digitale und schnell verfügbare Dienstleistungen steigt.
Wegen des Personalmangels und gleichzeitig wachsendem Leistungsanspruch rückt die automatisierte oder teilautomatisierte Bearbeitung von Verwaltungsprozessen in den Vordergrund.
Diese Unterstützung für die Sachbearbeiter ist keine Option, sondern eine Notwendigkeit um in den kommenden Jahre Handlungsfähig zu bleiben.

Ein Beispiel hierfür ist die Bearbeitung von Kindergeldanträgen bei der Familienkasse der Bundesagentur für Arbeit.
Mit einem jährlichen Aufkommen von mehreren Millionen Anträgen und starken saisonale Spitzen, etwa mit Beginn der Ausbildungen im Herbst, stößt die Familienkasse an ihre Kapazitätsgrenze.

% TODO: Faktencheck wie viele Anträge tatsächlich kommen

Um sich dieser Herausforderung zu stellen, wurde bereits im Oktober 2025 ein erstes teilautomatisiertes System produktiv gesetzt.
Dieses System basiert zu einem großen Teil auf einer Kombination aus \gls{OCR}-Erkennung und Objektklassifizierung mittels \gls{YOLO}-Modellen.

Obwohl dieses System einen wichtigen Schritt macht, zeigen sich während dem operativen Einsatz und der Entwicklung schon erste Grenzen.
Besonders die Erkennung der handschriftlich ausgefüllten Dokumente stellen für diese Pipeline eine enorme Herausforderung dar.

Genau an diesem Punkt setzt die Thesis an.
Durch den rasanten Fortschritt im Bereich der Generativen künstlichen Intelligenz öffnen sich neue Lösungen für Bearbeitung der Anträge.
Insbesondere \gls{VLM}s, die visuelle Informationen und Texte simultan verarbeiten, stellen eine vielversprechende Möglichkeit dar, die \gls{YOLO}-\gls{OCR}-Pipeline zu ersetzen.
Sie können flexibel auf unbekannte Layouts reagieren, handschriftlichen Text extrahieren und den Kontext interpretieren.

Mit der Entwicklung der \gls{VLM}s ergibt sich die Chance auf eine weitere Verbesserung für die Sachbearbeiter, jedoch muss getestet werden, ob diese Technologie bereits reif genug ist, um im öffentlichen Sektor eingesetzt zu werden.


\section{Problemstellung}\label{sec:problem}

\section{Zielsetzung}\label{sec:goal}

- Forschungsfragen
