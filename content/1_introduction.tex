\chapter{Einleitung}\label{ch:introduction}


\section{Motivation}\label{sec:motivation}

Die öffentliche Verwaltung in Deutschland steht vor einer großen Herausforderung.
Durch den demografischen Wandel verliert der öffentliche Sektor eine hohe Anzahl an erfahrenen Sachbearbeitern, während der Anspruch der Bürger steigt.
Dieser Wandel wird ohne Digitalisierung und Automatisierung der Prozesse kaum abzufangen sein.

Auch in der Bundesagentur für Arbeit (\gls{BA}) macht sich dieser Wandel, hin zu mehr Digitalisierung, bemerkbar.
Ein Beispiel hierfür ist die Bearbeitung von Kindergeldanträgen.
Mit einem jährlichen Aufkommen von mehreren Millionen Anträgen\cite{AnzahlAntraegeProJahr} und saisonalen Spitzen, kommen hier die Sachbearbeiter an ihre Grenzen.

Um sich dieser Herausforderung zu stellen, wurde bereits im Oktober 2025 ein teilautomatisiertes System produktiv gesetzt.
Hierbei wird sich auf die Kindergeldanträge von volljährigen Auszubildenden konzentriert.
Es werden Dokumente klassifiziert und mithilfe von \gls{YOLO}- und \gls{OCR}-Modellen Inhalte aus hochgeladenen Dokumenten erkannt und dem Sachbearbeiter als Vorschläge angezeigt.
Diese aktuelle \gls{YOLO}/\gls{OCR}-Pipeline wertet die Bearbeitung der Kindergeldanträge für die Sachbearbeiter bereits auf, stößt jedoch an ihre Grenzen.

Aufgrund der rasanten Entwicklung im Bereich der Vision Language Models (\glspl{VLM}) werden diese zunehmend interessanter.
In der vorliegenden Arbeit soll evaluiert werden, inwiefern ein solches Modell Potenzial hat, die \gls{YOLO}/\gls{OCR}-Pipeline zu ersetzen.


\section{Problemstellung}\label{sec:problem}

Seitdem die aktuelle Pipeline im operativen Einsatz ist, zeigen sich verschiedene Herausforderungen.

Das Hauptproblem ist die hohe Varianz der hochgeladenen Dokumente, insbesondere der Ausbildungsverträge.
Die verschiedenen Firmen und Kammern nutzen alle unterschiedliche Layouts, wodurch das Training der Modelle komplex ist.
Es ist schwierig, die Varianz in den Trainingsdaten abzubilden, zudem es zeit- und ressourcenintensiv ist, eine solche Menge an Trainingsdaten bereitzustellen.

Ein weiteres Problem ist die Qualität der hochgeladenen Dokumente.
Die Dokumente, die relevant für den Kindergeldantrag bei volljährigen Auszubildenden sind, benötigen in allen Fällen eine Unterschrift.
Dadurch werden die Dokumente zwingenderweise ausgedruckt, ausgefüllt und unterschrieben, um schlussendlich wieder eingescannt zu werden.
Dieser Prozess, den man auch Medienbruch nennt, hat eine schlechte Qualität zur Folge.

Angesichts der Verwendung von vielen unterschiedlichen Modellen ist die Wartung des Systems aufwändig.
Eine Anpassung an neue Gegebenheiten ist ein zeitaufwändiger Prozess.
Schon kleine Änderungen an der Laufzeitumgebung, wie zum Beispiel eine neue Python-Version, erfordern meistens ein erneutes Training der Modelle.


\section{Zielsetzung}\label{sec:goal}

Das Ziel dieser Bachelorarbeit ist die Entwicklung und Evaluation eines prototypischen Systems zur teilautomatisierten Klassifikation und Informationsextraktion von Dokumenten.
Angesichts der hohen Anzahl manuell bearbeiteter Anträge soll untersucht werden, inwieweit \glspl{VLM} diesen Prozess effizienter gestalten können.

Es wird eine Pipeline implementiert, die in der Lage ist, gescannte Dokumente als Bilddaten zu verarbeiten.
Das System soll den Dokumententyp eigenständig erkennen und definierte Inhalte, darunter handschriftliche Merkmale wie Unterschriften oder Stempel, in ein standardisiertes Format überführen.

Ein weiteres Ziel ist der Vergleich unterschiedlicher Modellansätze hinsichtlich ihrer Extraktionsperformance und Effizienz.
Hierbei wird ein kleineres, domänenspezifisch nachtrainiertes Modell gegen leistungsstärkere Modelle mit höherer Parameteranzahl antreten.
Es soll ermittelt werden, ob durch Fine-Tuning mit einem begrenzten Datensatz vergleichbare oder bessere Ergebnisse erzielt werden können als durch den Einsatz größerer Basismodelle.

Diese Arbeit konzentriert sich auf die Machbarkeit und die Evaluation der Modelle anhand eines Testdatensatzes.
Die Entwicklung zielt auf einen funktionsfähigen Prototyp ab, der lokal betrieben wird.
Eine vollständige Integration in das bestehende operative Fachverfahren ist nicht Gegenstand dieser Arbeit.