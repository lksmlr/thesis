\thispagestyle{empty}
\section*{Kurzdarstellung}\label{sec:kurzdarstellung}

Angesichts des demografischen Wandels und steigender Antragszahlen ist die Automatisierung von Verwaltungsprozessen für die Bundesagentur für Arbeit unverzichtbar.
Das bestehende System zur teilautomatisierten Verarbeitung von Ausbildungsnachweisen, basierend auf einer OCR/YOLO-Pipeline, weist jedoch Defizite bei variablen Dokumentenlayouts auf.

Ziel dieser Arbeit ist die Evaluation multimodaler Large Language Models als potenzielles Ersatzsystem.
Verglichen wurden die Modelle Pixtral-12B, Qwen-2.5-VL-7B und Qwen-2.5-VL-32B hinsichtlich ihrer Klassifikations- und Extraktionsleistung, Latenz sowie ihrer Ressourceneffizienz (Energieverbrauch und VRAM-Auslastung).
Ein Schwerpunkt lag auf der Untersuchung, ob ein kleineres, domänenspezifisch trainiertes Modell mit leistungsstärkeren Basismodellen konkurrieren kann.

Die Ergebnisse belegen, dass die bisherige Pipeline bei komplexen Dokumenten einen niedrigen F1-Score aufweist.
Das Qwen-2.5-VL-32B liefert die qualitativ besten Ergebnisse, ist jedoch aufgrund der Ressourcenineffizienz ungeeignet.
Als Handlungsempfehlung wird der Einsatz des nachtrainierten Qwen-2.5-VL-7B ausgesprochen.
Mit der Leistungssteigerung bietet dieses Modell den besten Kompromiss zwischen Leistungsfähigkeit und Ressourceneffizienz.
