\newglossaryentry{LLM}{
    name={LLM},
    description={Large Language Model}
}

\newglossaryentry{MLLM}{
    name={MLLM},
    description={Multimodal Large Language Model}
}

\newglossaryentry{VLM}{
    name={VLM},
    description={Vision Language Model}
}

\newglossaryentry{PEFT}{
    name={PEFT},
    description={Parameter-Efficient Fine-Tuning}
}

\newglossaryentry{YOLO}{
    name={YOLO},
    description={You Only Look Once}
}

\newglossaryentry{AuBe}{
    name={AuBe},
    description={Ausbildungsbescheinigungen}
}

\newglossaryentry{OCR}{
    name={OCR},
    description={Optical Character Recognition}
}

\newglossaryentry{FamKa}{
    name={FamKa},
    description={Familienkasse der Bundesagentur für Arbeit}
}

\newglossaryentry{JSON}{
    name={JSON},
    description={JavaScript Object Notation}
}

\newglossaryentry{VQA}{
    name={VQA},
    description={Visual Question Answering}
}

\newglossaryentry{LoRA}{
    name={LoRA},
    description={Low-Rank Adaptation}
}

\newglossaryentry{rsLoRA}{
    name={rsLoRA},
    description={Rank-Stabilized Low-Rank Adaptation}}

\newglossaryentry{IE}{
    name={IE},
    description={Information Extraction (\gls{IE}) umfasst nach Jurafsky und Martin die Aufgabe, Ereignisse oder Situationen in Dokumenten zu identifizieren und die entsprechenden Felder eines vorgegebenen Templates zu füllen\cite{IE}.
    Ziel hierbei ist es, vordefinierte Entitäten aus einem Dokumentenbild zu extrahieren und diese in ein maschinenlesbares Schema zu überführen.}
}

\newglossaryentry{TP}{
    name={TP},
    description={True Positive}
}

\newglossaryentry{TN}{
    name={TN},
    description={True Negative}
}

\newglossaryentry{FP}{
    name={FP},
    description={False Positive}
}

\newglossaryentry{FN}{
    name={FN},
    description={False Negative}
}

% TODO: statischer und dynamischer VRAM
\newglossaryentry{VRAM}{
    name={VRAM},
    description={Video Random Access Memory}
}

\newglossaryentry{EXIF}{
    name={EXIF},
    description={Exchangeable Image File Format}
}

\newglossaryentry{IDE}{
    name={IDE},
    description={Integrated Development Environment}
}

\newglossaryentry{BA}{
    name={BA},
    description={Bundesagentur für Arbeit}
}

\newglossaryentry{ViT}{
    name={ViT},
    description={Vision Transformer}
}

\newglossaryentry{Unsloth}{
    name={Unsloth},
    description={Open-Source-Framework zum beschleunigten und speichereffizienten Fine-Tuning von Large Models, das speziell für Methoden des PEFT optimiert ist.}
}

\newglossaryentry{Levenshtein-Similarity}{
    name={Levensthein-Similarity},
    description={Die Levenshtein-Similarity ist ein aus der Levenshtein-Distanz abgeleiteter, normalisierter Ähnlichkeitswert zwischen zwei Zeichenketten.
Formal basiert sie auf der Levenshtein-Distanz, also der minimalen Anzahl von Einfügungen, Löschungen oder Ersetzungen, die nötig sind, um eine Zeichenkette in eine andere zu überführen\cite{gld}.}
}

\newglossaryentry{Accuracy}{
    name={Accuracy},
    description={Metrik, die den Anteil der insgesamt korrekten Vorhersagen an der Gesamtzahl der getroffenen Vorhersagen beschreibt.}
}

\newglossaryentry{Precision}{
    name={Precision},
    description={Gibt den Anteil der tatsächlich positiven Fälle an allen vom Modell als positiv klassifizierten Fällen an.}
}

\newglossaryentry{Recall}{
    name={Recall},
    description={Gibt den Anteil der korrekt als positiv erkannten Fälle an allen tatsächlich vorhandenen positiven Fällen an.}
}

\newglossaryentry{F1-Score}{
    name={F1-Score},
    description={Das harmonische Mittel aus Precision und Recall.}
}

\newglossaryentry{Latenz}{
    name={Latenz},
    description={Die Zeitdauer, die das System für die Inferenz einer einzigen Antwort benötigt.}
}

\newglossaryentry{VRAM-Inferenz}{
    name={VRAM-Auslastung},
    description={Der maximale Bedarf an Grafikspeicher der GPU, der während der Ausführung eines Modells gemessen wird.}
}

\newglossaryentry{R-K-F}{
    name={R-K-F-Formel},
    description={Strategie im Prompt Engineering zur strukturierten Anweisung eines LLMs. Dabei steht R für Rolle (Rolle, die das Modell einnehmen soll), K für Kontext (relevante Hintergrundinformationen) und F für Format (Vorgabe der gewünschten Ausgabestruktur).}
}


\newglossaryentry{Prompt Engineering}{
    name={Prompt Engineering},
    description={Ist ein Prozess, um die effektivste Anweisung für eine spezifische Aufgabe zu finden\cite{prompt_engineering}.}
}

\newglossaryentry{Zero-Shot Prompting}{
    name={Zero-Shot Prompting},
    description={Eine Methode, bei der dem Modell eine Aufgabe gestellt wird, ohne dass zusätzliche Beispiele im Prompt enthalten sind.}
}

\newglossaryentry{One-Shot Prompting}{
    name={One-Shot Prompting},
    description={Eine Technik, bei der dem Modell genau ein Beispiel für die Lösung im Prompt bereitgestellt wird.}
}

\newglossaryentry{Few-Shot Prompting}{
    name={Few-Shot Prompting},
    description={Ein Ansatz, bei dem dem Modell mehrere Beispiele im Prompt übergeben werden.}
}


\glsaddall