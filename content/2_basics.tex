\chapter{Theoretische Grundlagen}\label{ch:basics}


\section{Fachliche Grundlagen}\label{sec:domain}

\subsection{Der Ist-Zustand}\label{subsec:current}

Wie bereits in der Einleitung angedeutet, setzt sich das aktuelle System, im Folgenden unter dem Projektnamen \gls{AuBe} geführt, aus einer mehrstufigen Pipeline zusammen.
Der schematische Ablauf ist in Abbildung\ref{fig:aube} dargestellt und lässt sich in drei logische Phasen unterteilen: Eingangsverarbeitung, Klassifikation und Informationsextraktion.

\begin{figure}[!ht]
    \centering
    \includegraphics[width=\textwidth]{figures/aube}
    \caption{Schematischer Workflow der aktuellen AuBe-Pipeline}
    \label{fig:aube}
\end{figure}

Der Prozess beginnt mit dem Upload eines Dokuments durch den Kunden.
Ein vorgeschaltetes System nimmt diese Dateien entgegen, die als PDF oder Bildformate vorliegen können.
In diesem Schritt findet eine initiale \gls{OCR}-Verarbeitung statt.
Das Ergebnis, welches an die \gls{AuBe}-Pipeline übergeben wird, ist ein standardisiertes PDF/A-Dokument sowie der extrahierte Text.

Im ersten Schritt der eigentlichen \gls{AuBe}-Pipeline entscheidet ein Text-Klassifikator auf Basis des \gls{OCR}-Textes, um welche Art von Dokument es sich handelt.
Hierbei wird zwischen drei Kategorien unterschieden:

\begin{itemize}
    \item \textbf{KG5b:} Eine spezifische Arbeitgeberbescheinigung, die für das Kindergeld relevant ist.
    \item \textbf{Vertrag:} Ein klassischer Berufsausbildungsvertrag.
    \item \textbf{Sonstiges:} Alle Dokumente, die nicht den beiden oben genannten Typen entsprechen.
\end{itemize}

Eine detaillierte Beschreibung der Dokumentenarten erfolgt in Kapitel\ref{ch:data}.

Wird ein Dokument als „Sonstiges“ klassifiziert, endet die Verarbeitung an dieser Stelle.
Nur die für den Fachprozess relevanten Typen (KG5b und Vertrag) werden weiterverarbeitet.

Für die relevanten Typen wird das PDF/A in ein Bild konvertiert und einem \gls{YOLO}-Modell übergeben.
Dieses \gls{YOLO}-Modell zeichnet Bounding-Boxen um die relevanten Felder, wobei daraufhin, ein weiteres \gls{OCR}-Modell die Erkennung in diesen Boxen durchführt


\subsection{Rahmenbedingungen und Infrastruktur}\label{subsec:infrastructure}


Die Entwicklung und Evaluation der Modelle erfolgt unter strengen datenschutzrechtlichen Auflagen.
Da im Rahmen dieser Arbeit personenbezogene Echtdaten verarbeitet werden, ist eine Nutzung öffentlicher Cloud-API-Dienste (wie OpenAI oder Anthropic) ausgeschlossen.
Stattdessen wird eine vollständig isolierte On-Premise-Infrastruktur verwendet.

Die technische Architektur ist in Abbildung\ref{fig:infrastructure} schematisch dargestellt.

\begin{figure}[!ht]
    \centering
    \includegraphics[width=0.8\textwidth]{figures/infrastructure}
    \caption{Schematische Darstellung der Trainings- und Inferenzinfrastruktur}
    \label{fig:infrastructure}
\end{figure}

Das System basiert auf einem abgeschotteten Kubernetes-Cluster.
Der Zugang zur Entwicklungsumgebung erfolgt für den Entwickler ausschließlich über eine noVNC-Schnittstelle (browserbasierter Remote-Desktop).
Dies stellt sicher, dass keine sensiblen Daten die gesicherte Umgebung verlassen.

Die Orchestrierung der Machine-Learning-Workflows erfolgt über Kubeflow.
Kubeflow ist eine Open-Source-Plattform die speziell für das Entwickeln, Trainieren und Deployen von Machine-Learning-Modellen entwickelt wurde.

Für die rechenintensiven Aufgaben, insbesondere das Fine-Tuning und die Inferenz der \gls{VLM}s, stehen innerhalb des Clusters zwei NVIDIA A40 GPUs, mit jeweils 48 GB vRAM\cite{NVIDIAA40}, zur Verfügung.


\section{Vision-Language Models (VLMs)}\label{sec:vlms}

\subsection{Vom Sprachmodell zum multimodalen Modell}\label{subsec:mllm}

\subsection{Architekturkomponenten}\label{subsec:components}


\section{Vorstellung der Modelle}\label{sec:models}

\subsection{Qwen-2.5-VL}\label{subsec:qwen2.5vl}

\subsection{Pixtral-12B}\label{subsec:pixtral}


\section{Fine-Tuning}\label{sec:finetuning}

\subsection{Parameter-Efficient Fine-Tuning (PEFT)}\label{subsec:peft}


\section{Information Extraction und Metriken}\label{sec:ieandmetrics}

\subsection{Information Extraction}\label{subsec:ie}

\subsection{Metriken}\label{subsec:metrics}


\section{Verwandte Arbeiten}\label{sec:relatedwork}
